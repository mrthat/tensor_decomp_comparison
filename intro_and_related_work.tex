\chapter{Introduction}

%There are many applications where we need to analyze face.
%In computer graphics we want to synthesize face images, 
%to edit existing images (say, change person's expression in a photo).
%In human-computer interaction we want virtual avatars to copy the mimic
%of real face. It may be convenient to be able to log into one's account
%by just allowing the computer to scan one's face.
%Some of these applications are naturally 3-dimensional (image synthesis
%is rendering a 2D image of a 3D model),
%some of them may become 3-dimensional in a short period of time (
%since 3D technologies become more and more accessible,
%it might be that a 3D-scanner will be a part of standard laptop
%in the future, as a web-camera is now). 


%The mainstream(?) approach to solve these problems 
%is a statistical one: we acquire a large database of faces
%and learn a statistical model with it.
%Statistical models can be discriminative or generative,
%we are interested in generative models.

%In this thesis we study the influence of numerical methods


\chapter{Related Work}


%Blanz and Vetter introduced their famous model in ~\cite{bl_vet_2003}.
%They arranged all the data in a matrix and proposed using principal
%component analysis to learn the low-dimensional subspace.

%In order to decouple identity and expression, 
%Vasilescu and Terzopoulos in~\cite{vt_2002} considered a 3-dimensional
%array (\textit{tensor}) and applied the same technique
%to tensors (the idea originated in text analysis,
%where the same technique was used to decouple author 
%and style).


