\section{Differential-Geometric Newton Method}
\label{sec_dg_newton}

%// add method from \cite{IDLAVH09}
Another differential-geometric variant of Newton method
was proposed by Ishteva et al in \cite{IDLAVH09}.
It is a further development of the method proposed in 
\cite{absil_2009}, and it is aimed at finding zeros
of a vector field that satisfies certain symmetry condition.
The gradient of the cost functions that we used before
does not have this property, so it is necessary to
define another vector field that has zero at the optimal
for Tucker2 matrices.  

Let $A \in \R^{d_1 \times d_2 \times d_3}$. In order to avoid conf
We fix matrix $U$ to be identity matrix of size $d_1$ and define two matrix-valued functions to shorten the notation:
\begin{eqnarray}
 R_2 := V^T A_{(2)} ( W \otimes U) \\
 R_3 := W^T A_{(3)} ( U \otimes V) 
\end{eqnarray}
Here $V$ and $W$ are (unknown) factor matrices, as before.
The vector field $\Phi$ in this method consists of two components
$\Phi_2$ and $\Phi_3$ defined by

\begin{eqnarray}
 \Phi_2 = V R_2 R_2^T - A_{(2)} ( W \otimes U)(W \otimes U)^T A_{(2)}^T V, \\
 \Phi_3 = W R_3 R_3^T - A_{(3)} ( U \otimes V)(U \otimes V)^T A_{(3)}^T W. 
\end{eqnarray}

The optimal matrices $V$ and $W$ are zeros of $\Phi$ (see the paper for the proof),
but $\Phi$ has other zeros which are not optimal, so we have to solve
the following system of equations iteratively with 
initial values that are close enough to optimum:
\begin{eqnarray}
    \Phi_2(V, W) = 0 \\
    \Phi_3(V, W) = 0
\end{eqnarray}

The function $\Phi$ has the desired symmetry property: 

\begin{eqnarray}
\forall S_2 \in O(d_2) \quad \Phi_2(VS_2) = \Phi_2(V) S_2, \\
\forall S_3 \in O(d_3) \quad \Phi_3(WS_3) = \Phi_3(W) S_3. 
\end{eqnarray}
Here $O(d_i)$ denotes the group of orthogonal $d_i \times d_i$ matrices. This implies
\begin{equation}
\forall S_2 \in O(d_2), S_3 \in O(d_3) \quad \Phi(VS_2, WS_3)= 0 \mbox{ if and only if } \Phi(V, W) = 0.
\end{equation}

This means that zeros 
of $\Phi$ are not isolated. This degeneracy will disappear, if
we identify $V$ with all $VS_2$
and $W$ with all $WS_3$. As in the previous section, the actual manifold
on which the algorithm works will be a direct product of two quotient manifolds.



Let us describe this quotient more formally.  
By $\R^{n\times p}_{*}$ we denote the set of all real maximal-rank $n \times p$ matrices.
\begin{equation}
    \R^{n \times p}_{*} = \{ A \in \R^{n \times p} : rank(A) = \min(n,p) \}.
\end{equation}
We always assume that $n \geq p$, so the rank is $p$. This condition holds if and only if
the first $p$ singular values of $A$ are non-zero, so it is equivalent to $\det(A^TA) \neq 0$.
As an open subset of $\R^{n \times p}$, this set is naturally a manifold.
For $A \in \R^{n \times p}_{*}$ we denote by $[A]$ the set 
$\{ AS \mid S \in O(p) \}$ and we put
\begin{equation}
    \R^{n \times p}_{*} / O(p) := \{ [A] \mid A \in \R^{n \times p}_{*} \}.
\end{equation}
This set is endowed with the structure of a smooth manifold by the quotient manifold
theorem (add ref). Moreover, it is a base of a fiber bundle whose total space
is $\R^{n \times p}$ and whose fibers are diffeomorphic to $O(p)$.
There is a projection map
\begin{equation}
    \pi \colon \R^{n \times p}_{*} \to \R^{n \times p}_{*} / O(p)
\end{equation}
that takes each $A$ to its equivalence class $[A]$. For every point $[A]$ of $\R^{n \times p}_{*} / O_p$, its pre-image
$\pi^{-1}( [A] ) = [A]$ (considered as a subset of $\R^{n \times p}_{*}$) is a submanifold.
It is well-known that the tangent space to $O(p)$ at the identity is
the set of all skew-symmetric matrices. In order to get \textit{vertical space}
at $X \in \R^{n \times p}_{*}$, we just multiply it with $X$, so
\begin{equation}
    \mathcal{V}_X = T_X[X] = X \mathfrak{o}(p).
\end{equation}
If we define the Riemannian metric on $\R^{n \times p}_{*}$ as
\begin{equation}
    g_X(Z_1, Z_2) := \tr(Z_1^T P^p_X(Z_2) + Z_1^T X (X^T X)^{-2} X^T Z_2),
\end{equation}
where 
\begin{equation}
    P^p_X(Y) := Y - X(X^TX)^{-1}X^T Y,
\end{equation}
then the orthogonal projector along $\mathcal{V}_X$, that is, the projection to a horizontal space,
will be 
\begin{equation}
    P^h_X(Y) := Y - Y \skw ((X^TX)^{-1} X^T Y).
\end{equation}
Here $\skw(A) = ( 1 / 2) ( A - A^T)$.
The image of the projector is a \textit{horizontal space}
\begin{equation}
    \mathcal{H}_X = \im (P^h_X)
\end{equation}


Suppose that we have a function $G \colon \R^{n \times p}_{*} \to \R^{n \times p}$
that satisfies the symmetry condition $G(XS) = G(X)S$ for all orthogonal $S$.
If we interpret the codomain of $G$
as $T_X\R^{n \times p}$, we identify $G$ with a vector field
on $\R^{n \times p}_{*}$. If we project this vector field to horizontal space,
e.g., introduce $\bar{\xi}(X) = P^h_X(G(X))$, we obtain the so-called 
\textit{horizontal lift} of $G$. Horizontal space can be identified
with the tangent space to the base $\R^{n \times p}_{*} / O(p)$, hence
we obtain the well-defined  correspondence between symmetric vector fields
on $\R^{n \times p}_{*}$ and vector fields on the quotient manifold
\begin{equation}
( X \mapsto G(X) ) \mapsto (X \mapsto P^h_{X}(G(X))).
\end{equation}
This correspondence preserves zeros and removes degeneracy, and it makes sense
to apply Newton method to find a zero of $\bar{\xi}$.


There is a general scheme for Newton method on a manifold $M$ with affine connection $\nabla$ (Adler et al. \cite{adler_spine}).
Let $\xi$ be a vector field on $M$ and $x$ be a tentative approximate
solution of $\xi(x) = 0$. To improve $x$, we solve the system 
\begin{equation}
    J_{\xi}(x)[\eta] = - \xi_x,
\end{equation}
where $J_{\xi}(x)$ is an operator on $T_xM$ defined by $J_{\xi}(x)[\zeta] = \nabla_{\zeta} \xi$
and get the vector $\eta \in T_xM$. Covariant differentiation with 
affine connection $\nabla$ is a generalization (which has geometric sense) of
an ordinary differentiation in the Euclidean space, so this step is the 
same as in ordinary Newton method. Now we need some way to get 
a new point on $M$ given $x$ and $\eta$. In order to do this we assume
that there is a function $R \colon TM \to M$. Each point of the tangent bundle $TM$
has two components $(p, v)$, $p \in M$, $v \in T_pM$, so $R$ is a function
of two arguments, the first one being a point $p$ on $M$
and the second one being a vector from $T_pM$. We use this function
to get the next approximation
\begin{equation}
    x_{+} = R(x, \eta).
\end{equation}
Not every function $R$ can be used as a retraction, the following 
conditions must hold (by $R_p$ we denote the restriction of $R$ to $T_pM$,  so $R_p \colon T_pM \to T_pM$):
\begin{enumerate}
    \item $R$ is a smooth function;
    %\item For every $p \in M$, $R_p$ is defined on some open subset of $T_pM$ that contains $\vec{0}$;
    \item $R_p(v) = x$ if and only if $v = \vec{0} \in T_pM$;

    %\item $DR_p(\vec{0}) = id_{T_pM}$ (the differential of $R_p$ at $0$ is identity)
    \item If $p \in M$, we denote the zero vector of $T_pM$ as $0_p$. The tangent space to $TM$ at point $(p, 0_p)$
        is $T_pM \times T_pM$, because $T_pM$ is a vector space and a tangent space to $T_pM$
        at every point is identified with $T_pM$ itself ($T_{0_p}(T_pM) = T_pM$). Thus the
        differential of $R$ at the point $(p, 0_p)$ is a linear function $(DR(p, 0_p))(\cdot, \cdot)$
        of two vectors from $T_pM$. We require
        \begin{equation}
          (DR(p, 0_p))(u, v) = u + v, \quad (u, v) \in T_pM \times T_pM.
        \end{equation}
\end{enumerate}

The affine connection on $\R^{n \times p}_{*} / O(p)$ used in this method is 
\begin{equation}
    \nabla_{\eta_{\pi(X)}} \xi (X) := P^h_X D \bar{\xi}(X)[\eta_X]
\end{equation}
and the retraction is
\begin{equation}
    X_{+} = X + \eta_X.
\end{equation}

Since we work on the product of two quotient manifolds,
the unknown vector $\Delta$ has two components 
$\Delta = (\Delta_2, \Delta_3) \in \mathcal{H}_V \times \mathcal{H}_W$, and
the system to be solved is
\begin{equation}
\label{dg_newt_system}
\left\{
\begin{aligned}
    P^h_V D (P^h_V \Phi_2)(V, W)[\Delta] & = - P^h_V \Phi_2(V, W) \\
    P^h_W D (P^h_W \Phi_3)(V, W)[\Delta] & = - P^h_W \Phi_3(V, W) \\
\end{aligned}
\right.
\end{equation}
Formally this is a system on $\R^{n \times p}_{*}$, but it represents
a system on the quotient manifold lifted to the total space.


We start by computing the horizontal lift $P^h_V\Phi_2$ of the field $\Phi_2$.
\begin{equation}
\begin{split}
 P^h_V( \Phi_2 ) & = V R_2 R_2^T - A_{(2)} (W \otimes U) (W \otimes U)^T A_{(2)}^T V \\
&  + V \skw( (V^T V)^{-1} V^T A_{(2)} (W \otimes U)(W \otimes U)^T A_{(2)}^T V )
\end{split}
\end{equation}
Now we need to differentiate this expression. We shall use the following properties
of differential (\cite{matrix_cookbook}):
\begin{enumerate}
    \item Differentiation commutes with linear operators. If $K$ is a linear operator and $\Phi$, $G$ are
    arbitrary matrix function, then 
    \begin{equation}
    D(K(\Phi)) = K(D(\Phi)).
    \end{equation}
        In particular, 
        \begin{eqnarray}
        D(\Phi+G) = D(\Phi) + D(G),\\ 
        D(\Phi^T) = (D(\Phi))^T.
        \end{eqnarray}
    \item Differential of constant is $0$.
    \item Differential of identity. Here by $V$ we mean the function
        $ (V, W) \mapsto V$,
        $\R^{n_2 \times d_2}_{*}  \times \R^{n_3 \times d_3}_{*} \to \R^{n_2 \times d_2}_{*}$.
        \begin{eqnarray}
        D(V)[\Delta_2, \Delta_3] = \Delta_2, \\
        D(W)[\Delta_2, \Delta_3] = \Delta_3.
        \end{eqnarray}
    \item Leibniz formula (differential of product) holds for matrix multiplication
        and for Kronecker product (the order of factors matters). If $\Phi$, $G$ are matrix-valued
        functions, then
        \begin{eqnarray}
        D(\Phi G)[\Delta] = D(\Phi)[\Delta] G + \Phi (D(G)[\Delta]) \\
        D(\Phi \otimes G)[\Delta] = D(\Phi)[\Delta] \otimes G + \Phi \otimes D(G)[\Delta]
        \end{eqnarray}
        \item Differential of inversion. If $\Phi$ is a matrix-valued function, then
        \begin{equation}
        D(\Phi^{-1}) = - \Phi^{-1} D(\Phi) \Phi^{-1}
        \end{equation}
\end{enumerate}

Applying these rules and plugging in the definitions of $R_2$ and $R_3$, we obtain
\begin{equation}
    \label{diff_p_h_v}
\begin{split}
 D(P^h_V \Phi_2) [ \Delta ]  &= \Delta_2 R_2 R_2^T + V \Delta_2^T A_{(2)} (W \otimes U) R_2^T \\ 
  & + V V^T A_{(2)} D ( ( W \otimes U)(W \otimes U)^T ) [\Delta] A_{(2)}^T V \\
 & + V R_2 (W \otimes U)^T A_{(2)}^T \Delta_2  \\
& - A_{(2)} D ( ( W \otimes U)(W \otimes U)^T ) [\Delta] A_{(2)}^T V \\
& -  A_{(2)}  ( W \otimes U)(W \otimes U)^T A_{(2)}^T \Delta_2 \\
& + \Delta_2 \skw ( (V^T V)^{-1} R_2 R_2^T ) \\
& + V \skw ( - (V^T V)^{-1} ( \Delta_2^T V + V^T \Delta_2) (V^T V)^{-1} R_2 R_2^T) \\ 
& +  V \skw ( (V^T V)^{-1} ( \Delta_2^T A_{(2)} ( W \otimes U) R_2^T + R_2 (W \otimes U)^T A_{(2)}^T \Delta_2 ) ) \\
& +  V \skw ( (V^T V)^{-1} ( V^T A_{(2)} D ( ( W \otimes U)(W \otimes U)^T)[\Delta]  A_{(2)}^T V ) ) \end{split}
\end{equation}

It remains to expand
\begin{equation}
D( ( W \otimes U)(W \otimes U)^T ) [\Delta] = ( \Delta_3 \otimes U) (W \otimes U)^T  + ( W \otimes U )( \Delta_3 \otimes U)^T. 
\end{equation}
Note that $U$ is fixed to be identity, otherwise we would have $4$ terms here instead of $2$.
The last step is to apply $P^h_V$ to this long expression. We shall not give the final result,
because in a program this operation can be performed automatically.
However, we should note that $P^h_V(V \skw A) = 0$ for any matrix $A$ (of the appropriate size),
so the last $3$ terms in \eqref{diff_p_h_v} do not contribute to the final expression.

In the same way we compute the lift of $\Phi_3$

\begin{equation}
\begin{split}
 P^h_W \Phi_3 & = W R_3 R_3^T - A_{(3)} (U \otimes V) (U \otimes V)^T A_{(3)}^T W  \\
              &  + W \skw( (W^T W)^{-1} W^T A_{(3)} (U \otimes V)(U \otimes V)^T A_{(3)}^T W )
\end{split}
 \end{equation}
and its differential
\begin{equation}
\label{diff_p_h_w}
\begin{split}
 D(P^h_W \Phi_3) [ \Delta ] & =\Delta_3 R_3 R_3^T + W \Delta_3^T A_{(3)} (U \otimes V) R_3^T  \\
&  W W^T A_{(3)} D ( ( U \otimes V)(U \otimes V)^T ) [\Delta] A_{(3)}^T W \\
& W R_3 (U \otimes V)^T A_{(3)}^T \Delta_3  \\
&- A_{(3)} D ( ( U \otimes V)(U \otimes V)^T ) [\Delta] A_{(3)}^T W \\
& -  A_{(3)}  ( U \otimes V)(U \otimes V)^T A_{(3)}^T \Delta_3 \\
& + \Delta_3 \skw ( (W^T W)^{-1} R_3 R_3^T ) \\
& + W \skw ( - (W^T W)^{-1} ( \Delta_3^T W + W^T \Delta_3) (W^T W)^{-1} R_3 R_3^T) \\
& +  W \skw ( (W^T W)^{-1} ( \Delta_3^T A_{(3)} ( U \otimes V) R_3^T + R_3 (U \otimes V)^T A_{(3)}^T \Delta_3 ) ) \\
& +  W \skw ( (W^T W)^{-1} ( W^T A_{(3)} D ( ( U \otimes V)(U \otimes V)^T)[\Delta]  A_{(3)}^T W ) ). \end{split}
\end{equation}

Here
\begin{equation}
 D( ( U \otimes V)(U \otimes V)^T ) [\Delta] = ( U \otimes \Delta_2 )(U \otimes V)^T + ( U \otimes V)  ( U \otimes \Delta_2 )^T 
\end{equation}


As before, the last $3$ terms in \eqref{diff_p_h_w} can be ignored. All that remains to do now
is to apply the same vectorization rules as in Newton-Grassmann method to get the matrix of the linear system.




\begin{algorithm}
\caption{Differential-geometric Newton method}\label{dgn_algo}
\begin{algorithmic}[1]
\Function{DiffGeomNewton} { $\A \in \R^{d_1 \times d_2 \times d_3}$, $r_2$, $r_3$}
\State $F_2, F_3 := \mbox{HOSVD}(\A, r_2, r_3)$
\Repeat 

\State Solve the system \eqref{dg_newt_system} to get $\Delta_2$, $\Delta_3$.
\State Update $F_2$ and $F_3$ by adding the $\Delta_i$ from the system, $F_i = F_2 + \Delta_i.$
    
\Until termination criterion satisfied or maximum iterations exhausted.


\State \Return $F_2$, $F_3$


\EndFunction
\end{algorithmic}
\end{algorithm}
